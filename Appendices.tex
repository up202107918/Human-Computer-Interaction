\chapter{Appendices}

\section{Information about fire vehicles and procedures during wildfires}
To better understand how a firefighter operates, who is the 
first line of defence, vehicles and people, we collected 
information from volunteer firefighters. \par
Firefighters utilise a range of forest firefighting vehicles, 
which are categorised according to their specific function and 
capacity. To illustrate, the VTTUs are tank vehicles with a 
capacity of 12,000, 15,000 or 16,000 litres. These vehicles 
supply the smaller vehicles that combat fires, such as the 
VFCI (with a capacity of 3,000 litres) or VLCI. The VOPE is a 
reconnaissance vehicle that is used to observe and analyse 
the situation in order to establish or adjust the firefighting
strategy. VECI vehicles are all-terrain vehicles and are the 
first to leave to fight the fire. \par
Upon reaching the ground, the firefighters establish 
communication with SIRESP, providing the coordinates of the 
location to the Operational Coordination Centre (OCC). The OCC
is responsible for overseeing the entire situation and 
coordinating the deployment of additional resources or 
reallocation of existing ground, air, and human resources. \par
The information in question proved invaluable in the construction 
of the resource and log panels, as it enabled us to develop a 
comprehensive understanding of the subject matter. \\ 
\begin{table}[H]
    \centering 
\caption{Forest firefighting vehicles} 
\begin{tabular}{ |c|c|c|c| } 
\hline
\textbf{Type} & \textbf{Description} & \textbf{Brand} & \textbf{Model} \\ 
\hline
VECI & VECI02 & Man(SEC) & T19F 19343 4x4 \\
\hline 
VECI & VECI04 & Mercedes & BARIBI-1217-4X4 \\ 
\hline
VFCI & VFCI01 & Iveco & ML150E28WS \\ 
\hline 
VLCI & VLCI05 & Mercedes (SEC) & UNIMOG U5000 \\ 
\hline 
VLCI & VLCI07 & Mitsubishi (SEC) & L200 \\ 
\hline 
VOPE & VOPE08 & Toyota & Hilux \\ 
\hline 
VOPE & VOPE01 & Suzuki (Moto4) & LT-F250 \\ 
\hline 
VTTU & VTTU04 & Volvo & FH12 \\
\hline 
VTTU & VTTU06 & Volvo (SEC) & FL 10 4x2 \\ 
\hline
\end{tabular}
\end{table}
