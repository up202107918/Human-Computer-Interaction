\chapter{Section 2} \label{section2}
\section{Outline of Implemented Features}
\section{Metrics Used for Evaluation}
In order to evaluate the preliminary sketches that were produced, it was collectively determined that each sketch would be assessed according to four distinct criteria.
The criteria were defined as follows:
\begin{enumerate}
    \item Functionality;
    \item Intuitiveness or clarity;
    \item Organization;
    \item Attractiveness/Design;
\end{enumerate}
Once all members of the group had evaluated the sketches, we proceeded to construct a matrix based on the above-mentioned four criteria. In this manner, a ranking was established, thereby facilitating discussion and progression in the process with regard to the most promising sketches.
\section{Type of Evaluation and its Final Results}
\textbf{Cameras Panel}
\begin{itemize}
    \item Functionality: 
\begin{table}[H]
\caption{Functionality Results Table}
\begin{tabular}{l*{9}{c}}
Name & Sketch 1 & Sketch 2 & Sketch 3 & Sketch 4 
& Sketch 5 & Sketch 6 & Sketch 7 & Sketch 8 & Sketch 9 \\
\hline 
Helena & 2 & 4 & 2 & 4 & 4 & 4 & 4 & 4 & 4 \\
Tomás & 3 & 3 & 3 & 4 & 4 & 4 & 4 & 3 & 3 \\ 
Rui & 2 & 3 & 2 & 3 & 3 & 2 & 3 & 3 & 3 \\ 
Sérgio & 2 & 3 & 2 & 3 & 3 & 2 & 3 & 3 & 3 \\ 
Total & 10 & 14 & 9 & 14 & 14 & 14 & 15 & 14 & 13 \\
\end{tabular}
\end{table}
\item Intuitiveness: 
\begin{table}[H]
\caption{Intuitiveness Results Table}
\begin{tabular}{l*{9}{c}}
    Name & Sketch 1 & Sketch 2 & Sketch 3 & Sketch 4 
    & Sketch 5 & Sketch 6 & Sketch 7 & Sketch 8 & Sketch 9 \\
    \hline 
    Helena & 3 & 4 & 2 & 4 & 3 & 3 & 4 & 4 & 4 \\
    Tomás & 2 & 3 & 2 & 4 & 3 & 3 & 4 & 3 & 3 \\ 
    Rui & 4 & 4 & 2 & 2 & 3 & 3 & 4 & 4 & 3 \\ 
    Sérgio & 2 & 3 & 2 & 4 & 3 & 2 & 2 & 3 & 3 \\ 
    Total & 11 & 14 & 8 & 14 & 12 & 11 & 14 & 14 & 13 \\
\end{tabular}
\end{table}
\item Organization:
\begin{table}[H]
    \caption{Organization Results Table}
    \begin{tabular}{l*{9}{c}}
        Name & Sketch 1 & Sketch 2 & Sketch 3 & Sketch 4 
        & Sketch 5 & Sketch 6 & Sketch 7 & Sketch 8 & Sketch 9 \\
        \hline 
        Helena & 3 & 4 & 3 & 3 & 4 & 4 & 4 & 4 & 4 \\
        Tomás & 2 & 3 & 2 & 4 & 3 & 3 & 4 & 3 & 3 \\ 
        Rui & 4 & 4 & 2 & 2 & 3 & 3 & 4 & 4 & 3 \\ 
        Sérgio & 2 & 4 & 2 & 4 & 3 & 4 & 4 & 4 & 3 \\ 
        Total & 11 & 15 & 9 & 13 & 13 & 14 & 16 & 15 & 13 \\
    \end{tabular}
\end{table}
\item Attractiveness: 
\begin{table}[H]
    \caption{Attractiveness Results Table}
    \begin{tabular}{l*{9}{c}}
        Name & Sketch 1 & Sketch 2 & Sketch 3 & Sketch 4 
        & Sketch 5 & Sketch 6 & Sketch 7 & Sketch 8 & Sketch 9 \\
        \hline 
        Helena & 2 & 4 & 1 & 3 & 4 & 3 & 3 & 3 & 4 \\
        Tomás & 2 & 3 & 1 & 3 & 4 & 3 & 4 & 3 & 3 \\ 
        Rui & 4 & 4 & 1 & 2 & 4 & 3 & 3 & 4 & 3 \\ 
        Sérgio & 2 & 3 & 1 & 4 & 3 & 2 & 3 & 3 & 3 \\ 
        Total & 10 & 14 & 4 & 12 & 15 & 11 & 13 & 13 & 13 \\
    \end{tabular}
\end{table}
\end{itemize}
With these results, we get the following scores: 
\begin{table}[H]
    \caption{Final Scores Table}
    \begin{tabular}{l*{9}{c}}
        Name & Sketch 1 & Sketch 2 & Sketch 3 & Sketch 4 
        & Sketch 5 & Sketch 6 & Sketch 7 & Sketch 8 & Sketch 9 \\
        \hline 
        Functionality & 10 & 14 & 9 & 14 & 14 & 14 & 15 & 14 & 13 \\
        Intuitiveness & 11 & 14 & 8 & 14 & 12 & 11 & 14 & 14 & 13 \\ 
        Organization & 11 & 15 & 9 & 13 & 13 & 14 & 16 & 15 & 13 \\ 
        Attractiveness & 10 & 14 & 4 & 12 & 15 & 11 & 13 & 13 & 13 \\ 
        Total & 42 & 57 & 30 & 53 & 54 & 50 & 58 & 56 & 52 \\
    \end{tabular}
\end{table}
\section{Iteration Process}
\section{How to Interact with the Graphical Interfaces Created}
