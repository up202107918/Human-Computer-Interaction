\chapter{Section 3}
\section{Summary of the Situation After the Assignment}
\section{Detected Problems and Sketches in Low Fidelity}
\section{Visual Features}
\subsection{Choice of Color Palette}
In order to create a colour palette for use in the 
prototype, the Coolors software was employed. Initially, 
19 colour palettes were generated (named palettes, 
where x was the number between 1-8, 10-17 and 19-21), 
which are presented beneath. \par 
Subsequently, each member of the group selected their 
five preferred palettes, as illustrated in the 
subsequent table.
\begin{table}[H]
    \caption{Selected Palletes by Each Element of the Group}
    \centerline{%
        \begin{tabular}{l|*5{c}}
            & \multicolumn{5}{c}{Selected Palletes} \\ 
            \hline 
            Helena & 1 & 5 & 6 & 10 & 20 \\
            Rui & 3 & 6 & 10 & 15 & 19 \\ 
            Sérgio & 4 & 5 & 10 & 16 & 19 \\ 
            Tomás & 3 & 5 & 10 & 15 & 20 \\
        \end{tabular}%
    }
\end{table}
\section{Improvements}
\section{Description of the Visible System Components}